\documentclass{article}
\usepackage{graphicx}
\usepackage{fullpage}
\usepackage{cite}
%\graphicspath{ {c:/users/administrator/desktop/} }

\title{CS 298 Proposal : EAP-SIM Vulnerabilities and Improvements}
\author{	
   Guided by Dr. Thomas Austin\\
   \\
   By : \\
   Akshay Baheti \\
   San Jos\'{e} State University \\
}

\begin{document}
\maketitle
\begin{abstract}
Today there is an the exponential increase in the number 3G/4G enabled devices. A cell
tower can support only a limited number of 3G/4G connection at a given time. This limitation of mobile technology leads to degraded service quality for customers at social gathering. WiFi is a solution to the problem as it can support large number of clients compared to Cellular data. Using WiFi to serve cellular clients is known as 3G Wifi Offloading[1]. Authentication over Wifi is a challenge though. This document describes the Extensible Authentication Protocol and several of its best-known security issues. This protocol is widely used for authentication over Wireless Networks. The document also introduces the basic functionality of EAP for Subscriber Identity Module(SIM) and discusses several of it's vulnerabilities. It later discusses possible improvements to avoid attacks of EAP-SIM.
\end{abstract}
\section{Project Deliverables}
\label{sec:Section 1}
The list of project deliverables includes the following
\begin{itemize}
 \item Detailed procedure for carrying out the attack
 \item wap-supplicant code with changes to showing high probability of the attack
 \item Server-side code with the defence of the attack
 \item Testcases and results showing improved protocol
\end{itemize}
\section{Challenging aspects of Project}
\label{sec:Section 2}
The main challenge involved is implementing an attack with a high success rate. The attack on EAP-SIM requires understanding the protocol and open source implementation of the same. Driving the defence requires a understanding both the client and server side implementation of the protocol. There are various open source implementations of the EAP protocol which vary significantly. Hence the attack implementation is a major challenge. 

Following the implementation find a suitable defence to the attack trying out the defence in the opensource is the next milestone for the project. Building a defence requires a through understanding of the protocol and the open source tool. Once a defence is ready ensuring that the defence does not significantly hurt the performance on the protocol remains a issue.

\section{Schedule}
\label{sec:Section 3}
\begin{table}[ht]
\caption{Timeline}
\centering 
\begin{tabular}{c c}
\hline \hline   
\\               
%\hline               
Week 1: May 4th - May 8th  & Understand the EAP­SIM protocol \\
\\
Week 2-3: May 11th - May 22nd  & Try to set up a basic freeradius setup for EAP SIM \\
\\
Week 4: May 25th - May 29th & Configure router for EAP­SIM to work with WiFi and free radius \\
\\
Week 5: June 1st - June 5th  & Configure opensource tool to run with EAP-SIM \\
\\
Week 6-7: June 8th - June 19th  & ­ Prepare for EAP-SIM DoS attack setup \\
\\
Week 8-9: June 22nd - July 3rd & Try out the EAP-SIM attack \\
\\
Week 10-11: July 6th - July 17th  & Find possible improvements/changes in protocol to avoid the attack \\ 
\\
Week 12-13: July 20th - July 31st & Find code point to try out the improvement \\
\\
Week 14-15: Aug 1st - Aug 15th  & Code the improvements in Hostapd \\
\\
Week 16-17: Aug 17th - Aug 28th & Code the improvements in wpasuplicant \\
\\
Week 18-19: Sept 1st - Sept 11th  &  Write test cases to test performance \\
\\
Week 20-21: Sept 14th - Sept 25th  & Carry out test to show performance of improvements \\ 
\\
Week 22-23: Sept 28th - Oct 19th &  Study other possible attacks on EAP \\
\\
Week 25: Oct 19th - Oct 23rd & Study other possible improvements on EAP \\
\\
Week 26-27: Oct 26th - Nov 6th & Write report \\
\\
Week 28-29: Nov 9th - Nov 20th  & Prepare for defence \\
\hline
\end{tabular}
\label{table:nonlin}
\end{table}
\clearpage
%\section{References}
%\label{sec:Section 4}
\bibliographystyle{plainurl}
\bibliography{milebib}

\end{document}