\documentclass{article}
\usepackage{graphicx}
\usepackage{fullpage}
\usepackage{cite}
%\graphicspath{ {c:/users/administrator/desktop/} }

\title{CS 298 Proposal : EAP-SIM Vulnerabilities and Improvements}
\author{	
   Guided by Dr. Thomas Austin\\
   \\
   By : \\
   Akshay Baheti \\
   San Jos\'{e} State University \\
}

\begin{document}
\maketitle
\begin{abstract}
Today there is an the exponential increase in the number 3G/4G enabled devices. A cell
tower can support only a limited number of 3G/4G connection at a given time. This limitation of mobile technology leads to degraded service quality for customers at social gathering. WiFi is a solution to the problem as it can support large number of clients compared to Cellular data. Using WiFi to serve cellular clients is known as 3G Wifi Offloading[1]. Authentication over Wifi is a challenge though. This document describes the Extensible Authentication Protocol and several of its best-known security issues. This protocol is widely used for authentication over Wireless Networks. The document also introduces the basic functionality of EAP for Subscriber Identity Module(SIM) and discusses several of it's vulnerabilities. It later discusses possible improvements to avoid attacks of EAP-SIM.
\end{abstract}
\section{Project Deliverables}
\label{sec:Section 1}
The list of project deliverables includes the following
\begin{itemize}
 \item Library supports a set of preconditions and postconditions that are specified in Java annotations
 \item Incorporates contract checking for objects of subclasses
 \item Support for writing contracts in scripting languages
 \item Support contracts for lambdas in Java 8
\end{itemize}
\section{Challenging aspects of Project}
\label{sec:Section 2}
The main challenge involved is implementing an attack with a high success rate. The attack on EAP-SIM requires understanding the protocol and open source implementation of the same. There are various open source implementations of the EAP protocol which vary significantly. Following the attack implementation a major challenge would be drive a basic defence for the attack. Driving the defence requires a understanding both the client and server side implementation of the protocol.

\section{Schedule}
\label{sec:Section 3}
\begin{table}[ht]
\caption{Timeline}
\centering 
\begin{tabular}{c c}
\hline \hline   
\\               
%\hline               
Week 1: May 4th - May 8th  & Understand the EAP­SIM protocol\\
\\
Week 2-3: May 11th - May 22nd  & Try to set up a basic freeradius setup for EAP SIM\\
\\
Week 4: May 25th - May 29th & Configure router for EAP­SIM to work with WiFi and free radius \\
\\
Week 5: June 1st - June 5th  & Configure mobile device to work with EAP­SIM \\
\\
Week 6-7: June 8th - June 19th  & ­ Buffer to get the SETUP ready\\
\\
Week 8-9: June 22nd - July 3rd & Understand the flaws in EAP SIM and work on the improvements \\
\\
Week 10-11: July 6th - July 17th  & Write the code to improvements in EAP­SIM\\ 
\\
Week 12-13: July 20th - July 31st & Work on code for EAP SIM \\
\\
Week 14-15: Aug 1st - Aug 15th  & Test out the improvements\\
\\
Week 16-17: Aug 17th - Aug 28th & Research on scripting languages to implement as conditions \\
\\
Week 18-19: Sept 1st - Sept 11th  & Implement contract conditions using Jython\\
\\
Week 20-21: Sept 14th - Sept 25th  & Implement contract conditions using Javascript\\
\\
Week 22-23: Sept 28th - Oct 9th & Implement contracts for lambdas in Java 8\\
\\
Week 24: Oct 12th - Oct 16th  & Implement contract conditions using Ruby\\
\\
Week 25: Oct 19th - Oct 23rd & Write test cases for the library\\
\\
Week 26-27: Oct 26th - Nov 6th & Write report\\
\\
Week 28-29: Nov 9th - Nov 20th  & Prepare for defense\\
\hline
\end{tabular}
\label{table:nonlin}
\end{table}
\clearpage
%\section{References}
%\label{sec:Section 4}
\bibliographystyle{plainurl}
\bibliography{milebib}

\end{document}