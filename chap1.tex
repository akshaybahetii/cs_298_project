\chapter{The Introduction}

This document presents an overview on some security issues that affect the Extensible Authentication Protocol.
This document introduces some basic concepts about EAP, its basic architecture and functionality. It later describes specific security flaws in a implementation of EAP -Subscriber Identity Module SIM. Finally is discusses some enchantments in EAP-SIM to improve the security of these protocols. 
 
\section{Extensible Authentication Protocol Overview} 

The Extensible Authentication Protocol (EAP) is an Internet standard that provides an infrastructure for network access clients and authentication servers. Extensible Authentication Protocol (EAP) enables the dynamic selection of the authentication mechanism at authentication time based on information transmitted in the Access-Request (that is, via RADIUS). Originally, EAP was created as an extension to PPP that allows for the development of arbitrary 
plug-in modules for current and future authentication methods and technologies [3110].Today, EAP is most often used in wireless LANs. Particularly, two wireless standards, WPA and WPA2, which have officially adopted several EAP 
methods as their main authentication mechanisms. 

Lately convergence of Wireless Local Area Networks (WLANs) and 3G systems is currently being deployed from the Third Generation Partnership Project (3GPP), which provides an interworking architecture as an add-on to the 3GPP cellular system specification. The integration of the two systems aims at combining them in such a way that brings out the functional advantages of each technology in a smooth way. The advantages of the cellular technology are related to the roaming capability, the subscription management and the authentication and key agreement procedure. On the other hand, WLANs provide better bandwidth and processing capabilities. To materialize this integration and ensure cooperation at the level of security, the EAP-SIM authentication and session key agreement protocol has been designed for the Global System for Mobile communications (GSM)/General Packet Radio Services (GPRS) network.

\begin{figure}[htb]
\centering	
\includegraphics[width=1\textwidth]{EAP_overview.jpg}
\caption{EAP Authentication Stack} 
\label{fig:EAP Authentication Stack}
\end{figure}

\section{EAP- Basics}

EAP was originally created as an extension to PPP to allow for the development of arbitrary network access authentication methods. With EAP, the specific authentication mechanism is not chosen during the link establishment phase of the PPP connection; instead, the PPP peers negotiate to perform EAP during the connection authentication phase. When the connection authentication phase is reached, the peers negotiate the use of a specific EAP authentication scheme known as an EAP method. After the EAP method is agreed upon, EAP allows for an open-ended exchange of messages between the access client and the authenticating server that can vary based on the parameters of the connection. The conversation consists of requests and responses for authentication information. The EAP method determines the length and details of the authentication conversation. Architecturally, an EAP infrastructure consists of the following:
EAP peer Computer that is attempting to access a network, also known as an access client. EAP authenticator An access point or network access server (NAS) that is requiring EAP authentication prior to granting access to a network. Authentication server A server computer that negotiates the use of a specific EAP method with an EAP peer, validates the EAP peer's credentials, and authorizes access to the network. Typically, the authentication server is a Remote Authentication Dial-In User Service (RADIUS) server. EAP is extensible through EAP methods that plug-in at both the EAP peer and authenticating server ends of a connection. To add support for a new EAP method, you install an EAP method library file on both the EAP peer and the authenticating server. This capability to extend EAP provides vendors with the opportunity to create new authentication schemes. EAP provides the highest flexibility to allow for more secure authentication methods. The EAP peer and the EAP authenticator send EAP messages using a supplicant-a software component that uses EAP to authenticate network access-and a data link layer transport protocol such as PPP or IEEE 802.1X. The EAP authenticator and the authentication server send EAP messages using RADIUS. The end result is that EAP messages are exchanged between the EAP components on the EAP peer and the authentication server. The following figure shows EAP infrastructure and information flow.



\section{The EAP protocol family}



